% This will be the main document for the Technical Networks paper to
% be written by the Eggnet team of Jordan Ell, Triet Huynh and Braden
% Simpson in association with Adrian Schroeter and Daniela Damian.

\documentclass[conference]{IEEEtran}

% Use of outside images
\usepackage{graphicx} 
% Use text inside euqations
\usepackage{amsmath}

\usepackage{balance}
\usepackage{float}
\floatstyle{plaintop}
\restylefloat{table}

% Correct bad hyphenation here
\hyphenation{op-tical net-works semi-conduc-tor}

% Begin the paper here
\begin{document}


% Paper title
% Can use linebreaks \\ within to get better formatting as desired
\title{Review: Exploring Multi-Threaded Java Application Performance on Multicore Hardware}

% Authors names
\author{\IEEEauthorblockN{Jordan Ell and Braden Simpson}
\IEEEauthorblockA{University of Victoria,
Victoria, British Columbia, Canada \\ jell@uvic.ca, braden@uvic.ca}
}

% Make the title area
\maketitle


\begin{abstract}
Abstract
\end{abstract}


\section{Problem}

The problems being discussed by the authors in this paper mainly focus on the lack of 
experimental knowledge in the domain of multicore hardware and interpreted languages.
The authors begin with a focus on the fact that multicore hardware systems are becoming
more and more common and continuing to expand in the number of cores and sockets moving
into the future due to Moore's Law. The authors also explain that multicore performance
for most standard programming languages (non-interpreted) is a much studied field and
has been extensively tested. However, the domain of interpreted languages is still a largely
unstudied and tested domain when it comes to performance on multicore multisocket hardware
systems. This can be attributed to the fact that instead of having just application threads to
worry about, you have virtual machine threads which can pause application execution, be needed
prior to certain execution points, and otherwise trigger non-deterministic actions of the various threads.

This paper presents their exploratory approach of the problem domain as a novel understanding of 
the field. This paper focuses on several factors of multicore hardware running an interpreted
program in Java which are the following: the clock frequency of different type of execution threads; 
the isolation of different types of execution threads between both cores and sockets in the 
hardware; the isolation of different types of threads between cores and socket at different
execution times in the program; the pinning of different thread types to specific cores; and some
combined affects of the tests listed above. The types of threads listed above which are mentioned
specifically in the problems to be tested and the results are: garbage collection, compilation, 
all JVM threads, and all application threads.

To the authors knowledge, most of these experimental tests have not been studied in the past
except for the clock frequency or power implications of their research. The authors explain
the importance that energy and power controls have, especially with regards to hand held devices.
The paper presents the idea of dark silicon which is the idea that high end servers will no longer
be able to power entire processors all the time. We believe this problem will soon be carried
over to the hand held market where battery life is becoming an increasing factor in the phone choice
of many consumers as phones become more powerful and require more energy to run. Intel's Turbo
Boost technology is a good representation of this problem as it allows for short duration 
clock frequency boosts on a limited number of cores.

In our opinion, the problems being investigated by this paper are extremely worthwhile when
it comes to anything with a limited processor speed or that is battery operated, and in the
case of mobile hardware, both. If the developers of Android can reduce power to an isolated
core on the phone which is in charge of running garbage collection to save 20\% processing
power, we believe this research will be of vital importance.



\bibliographystyle{IEEEtran}
\balance
\bibliography{paper}


% End of the paper
\end{document}
