% This will be the main document for the Technical Networks paper to
% be written by the Eggnet team of Jordan Ell, Triet Huynh and Braden
% Simpson in association with Adrian Schroeter and Daniela Damian.

\documentclass[conference]{IEEEtran}

% Use of outside images
\usepackage{graphicx} 
% Use text inside euqations
\usepackage{amsmath}

\usepackage{balance}
\usepackage{float}
\floatstyle{plaintop}
\restylefloat{table}

% Correct bad hyphenation here
\hyphenation{op-tical net-works semi-conduc-tor}

% Begin the paper here
\begin{document}


% Paper title
% Can use linebreaks \\ within to get better formatting as desired
\title{Paper Title}

% Authors names
\author{\IEEEauthorblockN{Jordan Ell and Braden Simpson}
\IEEEauthorblockA{University of Victoria,
Victoria, British Columbia, Canada \\ jell@uvic.ca, braden@uvic.ca}
}

% Make the title area
\maketitle

\begin{abstract}
Abstract
\end{abstract}


\section{Introduction}

\section{Discussion of Results}
The paper had some clear, well defined and well presented results.  Since the authors used a two-socket system, shown in figure~\ref{fig:cores}, they were able to many isolation tests between sockets.  Below some of the most interesting findings are discussed: \\  %TODO UPDATE THIS WITH JORDAN's LABEL NAME

\subsection{Isolating garbage collection threads to a separate socket}
 
The authors found that this leads to ``small performance degradation (no more than 17\%) for most benchmarks because of increased latency between sockets; however, one benchmark substantially benefits (66\%) from increased cache capacity."

This is interesting because if they separate the GC onto a different socket, if the process has a lot of cache allocation then we will see a large increase in efficiency, conversely, if the application must collect frequently then this will show a large degradation.  This finding would be good to keep in mind for applications similar to ``avrora'', one of their test projects which is sensitive to thread-core mapping.

\subsection{Isolating all threads or compilation thread to one socket in a power-constrainged environment}

The authors find ``When power-constrained in a multi-socket en- vironment, it is better to either keep application and JVM service threads on one socket, and power down the other socket(s), or to isolate the compilation thread onto the sec- ond socket and lower its frequency''. This is a really interesting stat, for two reasons.  The first reason is that if you are 


\section{Experiments}

\section{Recommendations}
The authors had a great theme of focusing on performance as a function of the power used, although they should have kept more along this theme, as well, we believe that there should be more integration of findings. 

\section{Appendix}

% End of the paper
\end{document}
